%%
%% Beginning of file 'sample62.tex'
%%
%% Modified 2018 January
%%
%% This is a sample manuscript marked up using the
%% AASTeX v6.2 LaTeX 2e macros.
%%
%% AASTeX is now based on Alexey Vikhlinin's emulateapj.cls 
%% (Copyright 2000-2015).  See the classfile for details.

%% AASTeX requires revtex4-1.cls (http://publish.aps.org/revtex4/) and
%% other external packages (latexsym, graphicx, amssymb, longtable, and epsf).
%% All of these external packages should already be present in the modern TeX 
%% distributions.  If not they can also be obtained at www.ctan.org.

%% The first piece of markup in an AASTeX v6.x document is the \documentclass
%% command. LaTeX will ignore any data that comes before this command. The 
%% documentclass can take an optional argument to modify the output style.
%% The command below calls the preprint style  which will produce a tightly 
%% typeset, one-column, single-spaced document.  It is the default and thus
%% does not need to be explicitly stated.
%%
%%
%% using aastex version 6.2
\documentclass[modern]{aastex62}

%% The default is a single spaced, 10 point font, single spaced article.
%% There are 5 other style options available via an optional argument. They
%% can be envoked like this:
%%
%% \documentclass[argument]{aastex62}
%% 
%% where the layout options are:
%%
%%  twocolumn   : two text columns, 10 point font, single spaced article.
%%                This is the most compact and represent the final published
%%                derived PDF copy of the accepted manuscript from the publisher
%%  manuscript  : one text column, 12 point font, double spaced article.
%%  preprint    : one text column, 12 point font, single spaced article.  
%%  preprint2   : two text columns, 12 point font, single spaced article.
%%  modern      : a stylish, single text column, 12 point font, article with
%% 		  wider left and right margins. This uses the Daniel
%% 		  Foreman-Mackey and David Hogg design.
%%  RNAAS       : Preferred style for Research Notes which are by design 
%%                lacking an abstract and brief. DO NOT use \begin{abstract}
%%                and \end{abstract} with this style.
%%
%% Note that you can submit to the AAS Journals in any of these 6 styles.
%%
%% There are other optional arguments one can envoke to allow other stylistic
%% actions. The available options are:
%%
%%  astrosymb    : Loads Astrosymb font and define \astrocommands. 
%%  tighten      : Makes baselineskip slightly smaller, only works with 
%%                 the twocolumn substyle.
%%  times        : uses times font instead of the default
%%  linenumbers  : turn on lineno package.
%%  trackchanges : required to see the revision mark up and print its output
%%  longauthor   : Do not use the more compressed footnote style (default) for 
%%                 the author/collaboration/affiliations. Instead print all
%%                 affiliation information after each name. Creates a much
%%                 long author list but may be desirable for short author papers
%%
%% these can be used in any combination, e.g.
%%
%% \documentclass[twocolumn,linenumbers,trackchanges]{aastex62}
%%
%% AASTeX v6.* now includes \hyperref support. While we have built in specific
%% defaults into the classfile you can manually override them with the
%% \hypersetup command. For example,
%%
%%\hypersetup{linkcolor=red,citecolor=green,filecolor=cyan,urlcolor=magenta}
%%
%% will change the color of the internal links to red, the links to the
%% bibliography to green, the file links to cyan, and the external links to
%% magenta. Additional information on \hyperref options can be found here:
%% https://www.tug.org/applications/hyperref/manual.html#x1-40003
%%
%% If you want to create your own macros, you can do so
%% using \newcommand. Your macros should appear before
%% the \begin{document} command.
%%
% added by DH
\usepackage{xspace}
\usepackage{}

\newcommand{\numax}{\mbox{$\nu_{\rm max}$}\xspace}
\newcommand{\Dnu}{\mbox{$\Delta \nu$}\xspace}
\newcommand{\dnu}{\mbox{$\delta \nu$}\xspace}
\newcommand{\muHz}{\mbox{$\mu$Hz}\xspace}
\newcommand{\teff}{\mbox{$T_{\rm eff}$}\xspace}
\newcommand{\logg}{\mbox{$\log g$}\xspace}
\newcommand{\feh}{\mbox{$\rm{[Fe/H]}$}\xspace}
\newcommand{\msun}{\mbox{$\mathrm{M}_{\sun}$}\xspace}
\newcommand{\mearth}{\mbox{$\mathrm{M}_{\oplus}$}\xspace}

\newcommand{\rsun}{\mbox{$\mathrm{R}_{\sun}$}\xspace}
\newcommand{\hipparcos}{\textit{Hipparcos}\xspace}
\newcommand{\gaia}{\textit{Gaia}\xspace}


\newcommand{\vdag}{(v)^\dagger}
\newcommand\aastex{AAS\TeX}
\newcommand\latex{La\TeX}
\newcommand\kepler{\emph{Kepler}\,}
\newcommand\ktwo{\emph{K2}\,}

\definecolor{linkcolor}{rgb}{0.1216,0.4667,0.7059}


\usepackage{xcolor, fontawesome}
\definecolor{twitterblue}{RGB}{64,153,255}
\newcommand\twitter[1]{\href{https://twitter.com/#1 }{\textcolor{twitterblue}{\faTwitter}\,\tt \textcolor{twitterblue}{@#1}}}


%% Tells LaTeX to search for image files in the 
%% current directory as well as in the figures/ folder.
\graphicspath{{../notebooks/}}

%% Reintroduced the \received and \accepted commands from AASTeX v5.2
\received{January 1, 2019}
\revised{January 7, 2019}
\accepted{\today}
%% Command to document which AAS Journal the manuscript was submitted to.
%% Adds "Submitted to " the arguement.
\submitjournal{ApJ}

%% Mark up commands to limit the number of authors on the front page.
%% Note that in AASTeX v6.2 a \collaboration call (see below) counts as
%% an author in this case.
%
%\AuthorCollaborationLimit=3
%
%% Will only show Schwarz, Muench and "the AAS Journals Data Scientist 
%% collaboration" on the front page of this example manuscript.
%%
%% Note that all of the author will be shown in the published article.
%% This feature is meant to be used prior to acceptance to make the
%% front end of a long author article more manageable. Please do not use
%% this functionality for manuscripts with less than 20 authors. Conversely,
%% please do use this when the number of authors exceeds 40.
%%
%% Use \allauthors at the manuscript end to show the full author list.
%% This command should only be used with \AuthorCollaborationLimit is used.

%% The following command can be used to set the latex table counters.  It
%% is needed in this document because it uses a mix of latex tabular and
%% AASTeX deluxetables.  In general it should not be needed.
%\setcounter{table}{1}

%%%%%%%%%%%%%%%%%%%%%%%%%%%%%%%%%%%%%%%%%%%%%%%%%%%%%%%%%%%%%%%%%%%%%%%%%%%%%%%%
%%
%% The following section outlines numerous optional output that
%% can be displayed in the front matter or as running meta-data.
%%
%% If you wish, you may supply running head information, although
%% this information may be modified by the editorial offices.
\shorttitle{No Massive Companion to GJ\,1151}
\shortauthors{B. J. S. Pope et al.}
%%
%% You can add a light gray and diagonal water-mark to the first page 
%% with this command:
% \watermark{text}
%% where "text", e.g. DRAFT, is the text to appear.  If the text is 
%% long you can control the water-mark size with:
%  \setwatermarkfontsize{dimension}
%% where dimension is any recognized LaTeX dimension, e.g. pt, in, etc.
%%
%%%%%%%%%%%%%%%%%%%%%%%%%%%%%%%%%%%%%%%%%%%%%%%%%%%%%%%%%%%%%%%%%%%%%%%%%%%%%%%%

%% This is the end of the preamble.  Indicate the beginning of the
%% manuscript itself with \begin{document}.

\begin{document}

\title{No Massive Companion to the Low-Frequency, Coherent Radio-Emitting M~Dwarf GJ\,1151}

\correspondingauthor{Benjamin J. S. Pope \twitter{fringetracker}}
\email{benjamin.pope@nyu}

\author[0000-0003-2595-9114]{Benjamin J. S. Pope}
\affiliation{Center for Cosmology and Particle Physics, Department of Physics, New York University, 726 Broadway, New York, NY 10003, USA}
\affiliation{Center for Data Science, New York University, 60 Fifth Ave, New York, NY 10011, USA}
\affiliation{NASA Sagan Fellow}

\author[0000-0002-8171-8596]{Megan Bedell}
\affiliation{Center for Computational Astrophysics, Flatiron Institute, 162 Fifth Ave, New York, NY 10010, USA}

\author[0000-0002-7167-1819]{Joseph R. Callingham}
\affiliation{ASTRON, Netherlands Institute for Radio Astronomy, Oude Hoogeveensedijk 4, Dwingeloo, 7991 PD, The Netherlands}

\author[0000-0002-0872-181X]{Harish K. Vedantham}
\affiliation{ASTRON, Netherlands Institute for Radio Astronomy, Oude Hoogeveensedijk 4, Dwingeloo, 7991 PD, The Netherlands}

\author[0000-0001-5648-9069]{Timothy W. Shimwell}
\affiliation{ASTRON, Netherlands Institute for Radio Astronomy, Oude Hoogeveensedijk 4, Dwingeloo, 7991 PD, The Netherlands}

\author[0000-0003-0872-7098]{Adrian Price-Whelan}
\affiliation{Center for Computational Astrophysics, Flatiron Institute, 162 Fifth Ave, New York, NY 10010, USA}

\author{friends}
%% Note that the \and command from previous versions of AASTeX is now
%% depreciated in this version as it is no longer necessary. AASTeX 
%% automatically takes care of all commas and "and"s between authors names.

%% AASTeX 6.2 has the new \collaboration and \nocollaboration commands to
%% provide the collaboration status of a group of authors. These commands 
%% can be used either before or after the list of corresponding authors. The
%% argument for \collaboration is the collaboration identifier. Authors are
%% encouraged to surround collaboration identifiers with ()s. The 
%% \nocollaboration command takes no argument and exists to indicate that
%% the nearby authors are not part of surrounding collaborations.

%% Mark off the abstract in the ``abstract'' environment. 
\begin{abstract}
The recent detection of circularly polarized, long-duration ($>8$~hr) low-frequency radio emission from the M4.5~dwarf GJ\,1151 has been interpreted as a Jupiter-like electron cyclotron maser instability arising from a star-planet interaction. The existence or parameters of the proposed planets have not been determined. In order to search for any planets or binary companions, we have obtained new HARPS-N observations

Using these data we do not detect a clear planetary signal, and we put $99^{\text{th}}$-percentile upper limits on the mass of any companion to GJ\,1151 at 5.1~\mearth. 

We infer interpolated, high-SNR, high-resolution spectra of GJ\,1151.
\href{https://github.com/benjaminpope/video}{\color{linkcolor}\faGithub} % shamelessly borrowed from Luger!

\end{abstract}

%% Keywords should appear after the \end{abstract} command. 
%% See the online documentation for the full list of available subject
%% keywords and the rules for their use.
% \keywords{editorials, notices --- 
% miscellaneous --- catalogs --- surveys}


\section{Introduction} 
\label{sec:intro}

Exoplanet science has flourished over the last three decades. The number of known planets has doubled nearly every two years since 1995 \citep{Mamajek2016} and this accelerating rate of discovery is projected to continue for at least the next decade if current and upcoming space-based surveys deliver their expected results. However, despite extensive searches \citep[e.g.][]{2000ApJ...545.1058B,2013A&A...552A..65L,Lynch18}, neither exoplanets nor their host stars have been detected at radio frequencies, as the non-flaring emission of such systems has been too faint for current telescopes. The SKA-precursor LOFAR \citep[the LOw-Frequency ARray:][]{lofar} has unparalleled sensitivity at 150~MHz where these interactions are expected to emit significant radiation via the electron cyclotron maser instability \citep{2007P&SS...55..598Z}: with its orders-of-magnitude increase in sensitivity and survey speed, the detection of nearby stars and planets is a realistic prospect \citep{pope19}. 

High stellar UV flux and flaring are thought to pose serious problems for habitability of planets around M~dwarfs \citep{Shields16,2019AsBio..19...64T}, though this may be not sufficiently severe in comparison to the early Earth to prevent the emergence of life \citep{2019MNRAS.485.5598O}. The stellar wind potentially poses a more severe problem. Theoretical studies have disagreed on the extent to which a planetary magnetosphere provides protection for its atmosphere from stripping by the radiation or wind of the host star \citep[e.g.][]{2013ApJ...770...23Z,2016A&A...596A.111R,2017ApJ...844L..13G}. Star-planet magnetic interactions (SPMI) analogous to the electrodynamic interaction of Jupiter and Io are theorized to occur when the interaction with the magnetized stellar wind of the host star is sub-Alfv\'{e}nic, and the planet is magnetically connected to its star and their magnetospheres are not separated by shocks. This is thought to be the case for Proxima~Centauri~b \citep{2016ApJ...833L...4G} and the inner TRAPPIST-1 planets \citep{2017ApJ...843L..33G}. With stellar wind flux orders of magnitude higher than that received by Earth, this may be a leading-order effect for stripping otherwise-habitable exoplanets of their atmospheres. The energy flux to the stellar surface from such an SPMI may give rise to optical H$\alpha$ aurorae at the magnetic connection footprint on the star \citep{2019arXiv190701020S}, or to radio signals. The search for the associated radio aurorae from M~dwarf planets is therefore a key component of understanding their long term evolution and habitability \citep{2017ApJ...849L..10B,2018ApJ...854...72T,2019arXiv190607089V}, but observational signatures of this have not so far been detected \citep[e.g.][]{Lynch18,Lenc18}.

Rather than explicitly searching for radio emission from known exoplanet hosts, \citet{Callingham_2019} cross-matched sources identified by the LOFAR Two-meter Sky Survey \citep[LoTSS:][]{lotss} with nearby stellar sources found by \gaia, finding the great majority of matches to be chance associations. But by restricting the survey to variable and circularly-polarized emission, the rate of chance associations with background radio galaxies is dramatically reduced. Based on this restricted cross-match, Vedantham et al. (submitted) have detected the quiescent M4.5 dwarf GJ\,1151 at low radio frequencies with LOFAR that has properties that suggest the low-frequency radio emission is driven by a star-planet magnetic interaction. While M~dwarfs are known to flare at low frequencies \citep[e.g.][]{Villadsen19}, this emission lasts over 8\,h and is $64\pm6\%$ circularly polarised. Such emission can be generated by the electron cyclotron maser instability (ECMI) through the interaction of the star with a short ($\sim 1-5$\,day) period planet or a close stellar companion as seen in UV Ceti-like stars \citep{2017ApJ...836L..30L}.

Since the radio emission implies a potential planetary or sub-stellar companion, in this Letter we present and analyze HARPS-N \citep[High Accuracy Radial velocity Planet Searcher:][]{harpsn} observations of GJ\,1151 in order to search for radial velocity signals of the proposed companions. We do not detect any planets, but place strong upper limits of a few Earth masses on the $M\sin{i}$ of any possible companions, ruling out any short-period massive objects or close binary companion.  

\section{RV Data}
\label{sec:k2}

We obtained 20 epochs of observations of GJ\,1151 with HARPS-N from 2018-12-20 to 2019-02-27, as a Director's Discretionary Time program (Program ID: A38DDT2; PI: Callingham). While RVs were extracted from these using the standard HARPS pipeline, its performance on this M4.5 dwarf was very poor, resulting in a spurious RV scatter of several km/s. We therefore reprocessed these data using \textit{wobble} \citep{wobble}, a data-driven package which simultaneously non-parametrically constructs a stellar spectral template and telluric spectral components and uses these, rather than model spectral masks, to extract radial velocities. 

We found that the second epoch (2018-12-22) had a significantly higher extracted RV uncertainty than the others, and accordingly excluded this from the global \textit{wobble} model. In order to assess template-dependent systematic errors, we conducted a `leave-one-out' cross-validation, excluding one additional epoch at a time and rerunning \textit{wobble} to search for consistency between the outputs. As seen in Figure~\ref{xvalidation}, the different resulting time series are broadly consistent in their directions of deviation from the mean, with a scatter between them of order $\sim$ the quoted uncertainties. We therefore believe the uncertainties on the \textit{wobble} RVs are realistic but that they are also model-dependent systematics, and therefore likely correlated. 

\begin{figure}
\plotone{comparison_validation.pdf}

\caption{\label{xvalidation}
Leave-one-out cross-validation of \textit{wobble} RVs. One epoch at a time is left out of the global model, and the results of processing the remaining epochs are shown in different colours. There is overall consistency between the different time series, with a diversity of order $\sim$ the quoted uncertainty between the individual realizations. % redo for complete cross-validation and final epoch added in
}
\end{figure}

\section{Keplerian Inference}
\label{sec:joker}

We use \textit{the Joker} pipeline \citep{joker}, which is optimized for small numbers of irregularly-spaced observations, to fit Keplerian signals and infer posterior planet parameters. We allow an additional astrophysical jitter (white noise) term to vary with a lognormal prior on jitter $\ln {(s/(\text{m/s}))^2} \sim {N}(1,2)$ for GJ\,1151.

\begin{figure}
\plotone{gj1151b_samples_trend.pdf}
\caption{\label{jokermodel1151}
\textit{The Joker} posterior samples for GJ\,1151.
}
\end{figure}


\begin{figure}
\plotone{cornerplot_km_trend.pdf}

\caption{\label{cornerplot}
Cornerplot of posterior samples from \textit{The Joker} for GJ\,1151.
}
\end{figure}

\section{Discussion}
\label{sec:discussion}

\citet{2019arXiv190505900H} using \kepler estimate that M3-M5 dwarfs host $1.19^{+0.70}_{−0.49}$ planets per star with radii from $0.5-2.5 R_\oplus$ and periods from  0.5-10\,d, the detection of a planet is actually less important than the exclusion of \emph{non}-planetary models, such as emission from a stellar binary interaction. We can confidently say that there is no short-period stellar binary companion to GJ~1151, and that the posterior for the RV trend of $0.02 \pm 0.03$ m/$\text{s}^{-1}\text{d}^{-1}$ is consistent with zero at $1\,\sigma$. 

\section{Conclusions}
\label{sec:conclusions}


% (1) The star-planet interaction calculations provide a heauristic lower limit on the range of planetary radius (r) and period (p) that are concistent with the properties of the observed radio emission.

% (2) Heuristically speaking, you can trade off planetary radius (r) with orbital distance (d) to get the same radio flux-density. The B-field goes down as d^-2 (parker spiral) and the planetary cross section for the interaction goes as r^2. So the radio data are providing a lower limit on the ratio r/d.

% (3) Now the mass (m) goes as r^3 and period (p) goes as d^2/3, we have a lower limit on m/p^2 from the radio data.

% (4) Again heuristically speaking taking the whole range of possibilities for the precise nature of interaction and the various efficiencies, we can say that an Earth-like planet in a 1-5d period will do the job. 

The results of this analysis conclusively rule out binary companions or planets on short orbits more massive than Neptune as the origin of the radio signal from GJ\,1151. We nevertheless cannot rule out in either case planets less massive than a few Earth masses. 

Vedantham et al. (submitted) derive an approximate mass and period estimate for their planet candidate based on the scaling relations of the SPMI. A planet with a larger radius $r$ has a larger cross-section for wind interaction $\propto r^2$, while the B-field drops with semi-major axis $a$ as $\approx a^{-2}$, so that the radio flux only provides a lower limit on $r/d$. Given the mass scaling $\propto r^3$ and orbital period $\propto a^{\frac{2}{3}}$, the radio detection provides a lower limit on $m/p^2$. An Earth-mass planet in a $1-5$\,d orbit is consistent with the observed radio flux, and at a sufficiently short period even a planet with a mass around that of $\sim$Mercury would be sufficient. While the data presented in this Letter conclusively rule out any more massive companion, there is a substantial region of parameter space for these planets that cannot be excluded and the star-planet interaction hypothesis remains reasonable.

Because our target is so red, to achieve significantly higher precision than attained by HARPS would require moving to the infrared, using one of the IR precision RV instruments currently being commissioned, such as the Habitable-zone Planet Finder \citep[HPF:][]{hpf} or CARMENES \citep{carmenes}, by which GJ\,1151 is already subject to monitoring \citep{carmenesinput}.

\section*{Acknowledgements} % add your acknowledgements text here!

This work was performed in part under contract with the Jet Propulsion Laboratory (JPL) funded by NASA through the Sagan Fellowship Program executed by the NASA Exoplanet Science Institute. 

BJSP acknowledges being on the traditional territory of the Lenape Nations and recognizes that Manhattan continues to be the home to many Algonkian peoples. We give blessings and thanks to the Lenape people and Lenape Nations in recognition that we are carrying out this work on their indigenous homelands.
%

This research made use of NASA's Astrophysics Data System; the SIMBAD database, operated at CDS, Strasbourg, France; the IPython package \citep{PER-GRA:2007}; SciPy \citep{scipy}; and Astropy, a community-developed core Python package for Astronomy \citep{astropy}. Some of the data presented in this paper were obtained from the Mikulski Archive for Space Telescopes (MAST). STScI is operated by the Association of Universities for Research in Astronomy, Inc., under NASA contract NAS5-26555. Support for MAST for non-HST data is provided by the NASA Office of Space Science via grant NNX13AC07G and by other grants and contracts. 

%%%%%%%%%%%%%%%%%%%%%%%%%%%%%%%%%%%%%%%%%%%%%%%%%%

%%%%%%%%%%%%%%%%%%%% REFERENCES %%%%%%%%%%%%%%%%%%


\bibliography{ms}


%% This command is needed to show the entire author+affilation list when
%% the collaboration and author truncation commands are used.  It has to
%% go at the end of the manuscript.
%\allauthors

%% Include this line if you are using the \added, \replaced, \deleted
%% commands to see a summary list of all changes at the end of the article.
%\listofchanges

\end{document}

% End of file `sample62.tex'.
